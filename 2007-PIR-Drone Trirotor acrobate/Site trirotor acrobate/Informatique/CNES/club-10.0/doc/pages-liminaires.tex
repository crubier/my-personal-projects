% -*- mode: latex; tex-main-file: "club.tex" -*-
% $Id: pages-liminaires.tex,v 1.42 2005/03/11 09:52:58 chope Exp $
\TitreNote{Manuel d'utilisation de la biblioth�que CLUB}
\ProjetNote{CLUB}
\NumEd{6}
\DateEd{22/10/2000}
\NumRev{9}
\DateRev{04/03/2005}
\RefNote{ESPACE/MS/CHOPE/CLUB/MU/001}

\RedigePar{\begin{tabular}[t]{@{}l@{}}
                L. Maisonobe, O. Queyrut\\G. Prat, F. Auguie, S. Vresk
           \end{tabular}}{CS SI/Espace/FDS}
\ValidePar{G. Prat}{CS SI/Espace/FDS}
\PourAppli{C. Fernandez-Martin}{CS SI/Espace/FDS}

\Hierarchie{
\textsc{Communications et Syst�mes}\vspace{3mm}\\
\textsc{Syst�mes d'Informations} \vspace{2ex} \\
\textsc{Direction Espace}
}

\ResumeNote{Ce document d�crit la biblioth�que des CLasses Utilitaires
de Base \bibliotheque{club}. Cette biblioth�que fournit des classes
\langage{c++} implantant des utilitaires de gestion de cha�nes de
caract�res, de gestion de texte, de gestion de donn�es structur�es
dans des fichiers, de traitement d'erreurs, de gestion d'options et de
partage de m�moire.}

\MotsClefsNote{biblioth�que, \langage{c++}}

\GestionConf{Oui}
\RespGestionConf{\textsc{Luc Maisonobe - CSSI}}

\Modifications{3}{0}{25/06/99}{SCS/CLUB/MU/99-001, passage au format CS}
\Modifications{3}{1}{12/08/99}{SCS/CLUB/MU/99-001, documentation du
passage du code d'erreur dans le TamponPartage de la classe BaseErreurs}
\Modifications{3}{2}{15/10/99}{SCS/CLUB/MU/99-001, changement de
num�ro de r�vision}
\Modifications{3}{3}{30/03/00}{SCS/CLUB/MU/99-001, utilisation de
notation UML}
\Modifications{4}{0}{04/08/00}{SCS/CLUB/MU/99-001, ajout de la
documentation de l'utilitaire difference}
\Modifications{5}{0}{04/09/00}{SCS/CLUB/MU/2000-001, passage en
feuille de style notechope et utilisation de la STL dans CLUB}
\Modifications{5}{1}{23/09/00}{SCS/CLUB/MU/2000-001, utilisation de
libtool pour g�n�rer club sous forme partag�e}
\Modifications{6}{0}{22/10/00}{SCS/CLUB/MU/2000-001, description des
classes UniqDataFile, MadonaFile, XMLFile, StructureFile et DataFile}
\Modifications{6}{1}{21/11/00}{SCS/CLUB/MU/2000-001, description des
modifications de configuration entre les versions 8.0 et 8.1}
\Modifications{6}{2}{04/12/00}{SCS/CLUB/MU/2000-001, description des
modifications de configuration entre les versions 8.1 et 8.2, ajout de
la description de \texttt{club-config}}
\Modifications{6}{3}{04/04/01}{SCS/CLUB/MU/2000-001, description des
modifications entre les versions 8.2 et 9.0, �limination de la classe
Adressage, reconnaissance des r�els fortran, corrections}
\Modifications{6}{4}{22/06/01}{ESPACE/MS/CHOPE/CLUB/MU/001, description des
modifications entre les versions 9.0 et 9.1, ajout de la classe CallTrace,
corrections}
\Modifications{6}{5}{28/03/03}{ESPACE/MS/CHOPE/CLUB/MU/001, description des
modifications entre les versions 9.1 et 9.2, ajout de constructeurs,
destructeurs}
\Modifications{6}{6}{28/07/03}{ESPACE/MS/CHOPE/CLUB/MU/001, description des
modifications entre les versions 9.2 et 9.3}
\Modifications{6}{7}{05/12/03}{ESPACE/MS/CHOPE/CLUB/MU/001, description des
modifications entre les versions 9.3 et 9.4}
\Modifications{6}{8}{11/06/04}{ESPACE/MS/CHOPE/CLUB/MU/001, description des
modifications entre les versions 9.4 et 9.5}
\Modifications{6}{9}{04/03/05}{ESPACE/MS/CHOPE/CLUB/MU/001, description des
modifications entre les versions 9.5 et 10.0}

\CausesEvolution{annonce des modifications apport�es en version 10.0}

\DiffusionInterne[2 exemplaires]{CS SI}{ESPACE/FDS}{}

\DiffusionExterne[3 exemplaires]{CNES}{DCT/SB/MS}{}

