% -*- mode: latex; tex-main-file: "cantor.tex" -*-
% $Id: macros-bibliotheques.tex,v 1.6 2005/03/11 10:43:03 chope Exp $
\usepackage{longtable}

\usepackage[dvips]{changebar}
\setcounter{changebargrey}{40}
\outerbarstrue
\setlength{\changebarsep}{5pt}

\makeatletter
\renewcommand{\l@subsection}{\@dottedtocline{2}{1.5em}{2.7em}}
\makeatother

\newcommand{\fonc}[1]{\textbf{#1}}
\newcommand{\throw}[1]{\textbf{throw}(#1)}
\newcommand{\argument}[1]{\textsl{#1}}
\newcommand{\langage}[1]{\textsc{#1}}
\newcommand{\bibliotheque}[1]{\textsc{#1}}
\newcommand{\outil}[1]{\texttt{#1}}

\newcommand{\template}[2]{#1<\relax{}#2>\relax{}}

\newlength{\hauteurPile}\newlength{\hauteurPar}
\newcommand{\pileDescendante}[1]{\settoheight{\hauteurPile}{\shortstack[l]{#1}}
\settoheight{\hauteurPar}{()}\addtolength{\hauteurPile}{-\hauteurPar}
\raisebox{-\hauteurPile}{\shortstack[l]{#1}}}

\newlength{\largeurPar}
\newcommand{\signature}[2]{\settowidth{\largeurPar}{(}
\pileDescendante{{#1}\\\hspace{2em}\shortstack[l]{\hspace{-\largeurPar}#2}}}

\newlength{\largeurTableFonctions}

\newenvironment{tableFonctionsLibre}[3]{\begin{table}[htbp]\caption{#1}#2
\begin{center}\footnotesize\settowidth{\largeurTableFonctions}{#3}
\renewcommand{\arraystretch}{1.5}\begin{tabular}{|l|p{\largeurTableFonctions}|}
\hline \multicolumn{1}{|c|}{signature} & \multicolumn{1}{c|}{description}\\
\hline\hline\\}
{\hline\end{tabular}\end{center}\end{table}}

\newenvironment{tableFonctionsFixe}[3]{\footnotesize
\settowidth{\largeurTableFonctions}{#3}
\renewcommand{\arraystretch}{1.5}
\begin{longtable}{|l|p{\largeurTableFonctions}|}\caption{#1}#2\\
\hline\multicolumn{1}{|c|}{signature} & \multicolumn{1}{c|}{description}\\
\hline\hline\endfirsthead\caption[]{#1 (suite)}\\
\hline\multicolumn{1}{|c|}{signature} & \multicolumn{1}{c|}{description}\\
\hline\hline\endhead
\hline\multicolumn{2}{|r|}{� suivre ...}\\\hline\endfoot
\hline\endlastfoot}{\end{longtable}}

\newlength{\largeurTableAttributs}

\newenvironment{tableAttributsLibre}[3]{\begin{table}[htbp]\caption{#1}#2
\begin{center}\settowidth{\largeurTableAttributs}{#3}
\begin{tabular}[b]{|c|c|p{\largeurTableAttributs}|}
\hline nom & type & \multicolumn{1}{c|}{description}\\
\hline\hline}
{\hline\end{tabular}\end{center}\end{table}}

\newenvironment{tableAttributsFixe}[3]{
\settowidth{\largeurTableAttributs}{#3}
\begin{longtable}{|c|c|p{\largeurTableAttributs}|}\caption{#1}#2\\
\hline nom & type &\multicolumn{1}{c|}{description}\\\hline\hline\endfirsthead
\caption[]{#1 (suite)}\\
\hline nom & type &\multicolumn{1}{c|}{description}\\\hline\hline\endhead
\hline\multicolumn{3}{|r|}{� suivre ...}\\\hline\endfoot
\hline\endlastfoot}{\end{longtable}}
