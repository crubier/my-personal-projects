% -*- mode: latex; tex-main-file: "marmottes-utilisateur.tex" -*-
% $Id: SenseurVecteur.tex,v 1.11 2004/06/21 14:43:27 marmottes Exp $
\subsection{classe SenseurVecteur}\label{sec:SenseurVecteur}

\subsubsection*{description}\label{sec:SenseurVecteur-desc}

Cette classe implante les senseurs mesurant des angles entre
vecteurs. De tels senseurs repr�sentent par exemple les senseurs des
satellites spinn�s qui utilisent le mouvement du satellite sur
lui-m�me pour produire leur mesure. La classe sert �galement de classe
de base pour d�finir des pseudo-senseurs (SenseurCartesien,
SenseurElevation).
\subsubsection*{interface publique}\label{sec:SenseurVecteur-int}
\begin{verbatim}
#include "marmottes/SenseurVecteur.h"
\end{verbatim}

\begin{tableFonctionsFixe}{SenseurVecteur : m�thodes publiques}
{\label{tab:SenseurVecteur-met-pub}}
{force le senseur � convertir les unit�s de mesures dans ses sorties}


\signature{\fonc{SenseurVecteur}}
          {(const string\& \argument{nom},\\
            const RotDBL\& \argument{repere},\\
            const VecDBL\& \argument{axeCalage},\\
            double \argument{precision},\\
            codeCible \argument{code},\\
            const StationCible *\argument{ptrStation},\\
            const VecDBL\& \argument{observe},\\
            Parcelle *\argument{ptrChampDeVue},\\
            Parcelle* \argument{ptrChampInhibitionSoleil},\\
            Parcelle* \argument{ptrChampInhibitionLune},\\
            Parcelle* \argument{ptrChampInhibitionCentral},\\
            double \argument{margeEclipseSoleil},\\
            double \argument{margeEclipseLune},\\
            double \argument{seuilPhaseLune},\\
            const VecDBL\& \argument{reference})\\
            \throw{CantorErreurs}
          }&

construit une instance � partir des donn�es technologiques\\

\hline

\signature{\fonc{SenseurVecteur} (const SenseurVecteur\& \argument{s})}
          {}&

constructeur par copie\\

\signature{SenseurVecteur\& \fonc{operator =}}
          {(const SenseurVecteur\& \argument{s})}&

affectation\\

\hline

\signature{\fonc{\~{}SenseurVecteur} ()}
          {}&

destructeur, ne fait rien dans cette classe\\

\hline

\signature{void \fonc{respecterMesures} ()}
          {}&

force le senseur � respecter les unit�s de mesures dans ses sorties\\

\signature{void \fonc{convertirMesures} ()}
          {}&

force le senseur � convertir les unit�s de mesures dans ses sorties\\

\hline

\signature{Senseur* \fonc{copie} () const}
          {}&

op�rateur de copie virtuel\\

\hline

\signature{void \fonc{nouveauRepere} (const RotDBL\& \argument{nouveau})}
          {}&

remplace le rep�re du senseur par le \argument{nouveau}\\

\signature{void \fonc{modeliseConsigne}}
          {(const Etat\& \argument{etat}, double \argument{valeur})\\
          \throw{CantorErreurs, MarmottesErreurs}}&

mod�lise la consigne \argument{valeur} dans l'\argument{etat} fourni\\

\signature{double \fonc{mesure} (const Etat\& \argument{etat})}
          {\throw{MarmottesErreurs}}&

retourne la mesure que produirait le senseur dans l'\argument{etat}
fourni\\

\end{tableFonctionsFixe}

\subsubsection*{implantation}\label{sec:SenseurVecteur-impl}
Les attributs priv�s sont d�crits sommairement dans la
table~\ref{tab:SenseurVecteur-att-priv}, il n'y a pas d'attribut prot�g�.
\begin{tableAttributsFixe}{attributs priv�s de la classe SenseurVecteur}
{\label{tab:SenseurVecteur-att-priv}}
{vecteur de r�f�rence en rep�re satellite}

reference\_ & VecDBL & vecteur de r�f�rence en rep�re satellite\\

\end{tableAttributsFixe}

Les m�thodes prot�g�es sont d�crites dans la table~\ref{tab:SenseurVecteur-met-prot}.
\begin{tableFonctionsFixe}{SenseurVecteur : m�thodes prot�g�es}
{\label{tab:SenseurVecteur-met-prot}}
{constructeur par d�faut. Il est d�fini explicitement uniquement pour }

\signature{\fonc{SenseurVecteur} ()}
          {}&

constructeur par d�faut. Il est d�fini explicitement uniquement pour
pr�venir celui cr�� automatiquement par le compilateur et ne doit pas �tre
utilis�.
\\
\end{tableFonctionsFixe}
