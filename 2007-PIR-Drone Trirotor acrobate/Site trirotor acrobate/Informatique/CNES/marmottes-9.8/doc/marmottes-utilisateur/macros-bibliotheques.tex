% -*- mode: latex; tex-main-file: "marmottes-utilisateur.tex" -*-
% $Id: macros-bibliotheques.tex,v 1.14 2005/03/11 11:11:39 chope Exp $
\usepackage{longtable}

\usepackage[dvips]{changebar}
\setcounter{changebargrey}{40}
\outerbarstrue
\setlength{\changebarsep}{5pt}

\makeatletter
\renewcommand{\l@subsection}{\@dottedtocline{2}{1.5em}{2.7em}}
\makeatother

\newcommand{\fonc}[1]{\textbf{#1}}
\newcommand{\throw}[1]{\textbf{throw (#1)}}
\newcommand{\argument}[1]{\textsl{#1}}
\newcommand{\langage}[1]{\textsc{#1}}
\newcommand{\bibliotheque}[1]{\textsc{#1}}
\newcommand{\outil}[1]{\texttt{#1}}
\newcommand{\nomBloc}[1]{\texttt{\scriptsize #1}}

\newlength{\hauteurPile}\newlength{\hauteurPar}
\newcommand{\pileDescendante}[1]{\settoheight{\hauteurPile}{\shortstack[l]{#1}}
\settoheight{\hauteurPar}{()}\addtolength{\hauteurPile}{-\hauteurPar}
\raisebox{-\hauteurPile}{\shortstack[l]{#1}}}

\newlength{\largeurPar}
\newcommand{\signature}[2]{\settowidth{\largeurPar}{(}
\pileDescendante{{#1}\\\hspace{2em}\shortstack[l]{\hspace{-\largeurPar}#2}}}

\newlength{\largeurTableFonctions}

\newenvironment{tableFonctionsLibre}[3]{\begin{table}[htbp]\caption{#1}#2
\begin{center}\footnotesize\settowidth{\largeurTableFonctions}{#3}
\renewcommand{\arraystretch}{1.5}\begin{tabular}{|l|p{\largeurTableFonctions}|}
\hline \multicolumn{1}{|c|}{signature} & \multicolumn{1}{c|}{description}\\
\hline\hline\\}
{\hline\end{tabular}\end{center}\end{table}}

\newenvironment{tableFonctionsFixe}[3]{\footnotesize
\settowidth{\largeurTableFonctions}{#3}
\renewcommand{\arraystretch}{1.5}
\begin{longtable}{|l|p{\largeurTableFonctions}|}\caption{#1}#2\\
\hline\multicolumn{1}{|c|}{signature} & \multicolumn{1}{c|}{description}\\
\hline\hline\endfirsthead\caption[]{#1 (suite)}\\
\hline\multicolumn{1}{|c|}{signature} & \multicolumn{1}{c|}{description}\\
\hline\hline\endhead
\hline\multicolumn{2}{|r|}{� suivre ...}\\\hline\endfoot
\hline\endlastfoot}{\end{longtable}}

\newlength{\largeurTableAttributs}

\newenvironment{tableAttributsLibre}[3]{\begin{table}[htbp]\caption{#1}#2
\begin{center}\settowidth{\largeurTableAttributs}{#3}
\begin{tabular}[b]{|c|c|p{\largeurTableAttributs}|}
\hline nom & type & \multicolumn{1}{c|}{description}\\
\hline\hline}
{\hline\end{tabular}\end{center}\end{table}}

\newenvironment{tableAttributsFixe}[3]{
\settowidth{\largeurTableAttributs}{#3}
\begin{longtable}{|c|c|p{\largeurTableAttributs}|}\caption{#1}#2\\
\hline nom & type &\multicolumn{1}{c|}{description}\\\hline\hline\endfirsthead
\caption[]{#1 (suite)}\\
\hline nom & type &\multicolumn{1}{c|}{description}\\\hline\hline\endhead
\hline\multicolumn{3}{|r|}{� suivre ...}\\\hline\endfoot
\hline\endlastfoot}{\end{longtable}}

\newcounter{exemple}\newlength{\largeurexemple}
\newenvironment{exemple}[2]{\refstepcounter{exemple}\begin{center}
\settowidth{\largeurexemple}{\ttfamily#2}\begin{minipage}{\largeurexemple}
\centerline{Expl. \arabic{exemple} -- \textit{#1}}\ttfamily\raggedleft}
{\end{minipage}\end{center}}

\newlength{\largeurTableSenseurs}
\newenvironment{tableSenseur}[3]{
\settowidth{\largeurTableSenseurs}{#3}
\begin{table}[htbp]\caption{#1}\renewcommand{\arraystretch}{1.5}
\begin{center}\begin{tabular}[htbp]{|c|c|p{\largeurTableSenseurs}|}
\hline
nom du bloc & type & \multicolumn{1}{c|}{description} \\
\hline
\nomBloc{type} & mot-clef & Identificateur permettant de reconna�tre le
type de senseur, vaut obligatoirement #2. \\
\nomBloc{precision} & r�el & Pr�cision du senseur en degr�s, utilis�e
pour la convergence. \\
\nomBloc{repere} & rotation & D�finition du rep�re senseur, le plus
pratique est d'utiliser la syntaxe \texttt{\{i\{...\} j\{...\}
k\{...\}\}}, on donne alors les coordonn�es \emph{en rep�re satellite}
des axes canoniques du senseur. \\
\nomBloc{axe\_calage} & vecteur & Bloc \emph{optionnel} d�finissant un
axe de calage autour duquel le senseur peut tourner
(voir~\ref{sec:orientation}). \\}
{\hline\end{tabular}\end{center}\end{table}}

\newcommand{\entreesChamps}{

\nomBloc{champ\_de\_vue} & parcelle & Champ de vue du senseur \emph{en
rep�re senseur}.\\

\nomBloc{champ\_d\_inhibition\_corps\_central} 
& parcelle & Bloc \emph{optionnel} d�finissant le champ d'inhibition
du senseur par un astre de type corps central \emph{en rep�re senseur}. 
Vide par d�faut. \\

\nomBloc{champ\_d\_inhibition\_soleil} & parcelle & Bloc
\emph{optionnel} d�finissant le champ d'inhibition par le soleil du
senseur \emph{en rep�re senseur}. Vide par d�faut. \\

\nomBloc{marge\_eclipse\_soleil} & r�el & Bloc \emph{optionnel}
d�finissant la marge angulaire sur la suppression des inhibitions pour
les passages en �clipse. \\

\nomBloc{champ\_d\_inhibition\_lune} & parcelle & Bloc \emph{optionnel}
d�finissant le champ d'inhibition par la lune du senseur \emph{en
rep�re senseur}. Vide par d�faut. \\

\nomBloc{marge\_eclipse\_lune} & r�el & Bloc \emph{optionnel}
d�finissant la marge angulaire sur la suppression des inhibitions pour
les passages en �clipse. \\

\nomBloc{seuil\_phase\_lune} & scalaire & Bloc \emph{optionnel}
d�finissant le seuil sur l'angle Soleil/Satellite/Lune au
\emph{dessous} duquel la lune devient g�nante (0 signifie que la lune
n'est jamais g�nante, 180 signifie qu'elle est toujours g�nante).}

\newcommand{\entreesSenseurOptique}{

\nomBloc{cible} & mot-clef & Cible du senseur. La
table~\ref{tab:cibles}, page~\pageref{tab:cibles} donne la liste des
valeurs reconnues. \\

\nomBloc{observe} & vecteur & Coordonn�es de la direction
d'observation du senseur \emph{en rep�re senseur}, uniquement si
\texttt{cible} vaut \texttt{vitesse-sol-apparente} \\

\nomBloc{station} & structure & D�finition de la station observ�e,
uniquement si \texttt{cible} vaut \texttt{station}. \\

\entreesChamps}

\newenvironment{interface}[1]
{\begin{tabbing}\hspace*{10mm}\=\kill interface \langage{#1} :\+\\}
{\end{tabbing}}

