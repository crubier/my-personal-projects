% -*- mode: latex; tex-main-file: "marmottes-utilisateur.tex" -*-
% $Id: SenseurCartesien.tex,v 1.11 2004/06/21 14:43:06 marmottes Exp $
\subsection{classe SenseurCartesien}\label{sec:SenseurCartesien}

\subsubsection*{description}\label{sec:SenseurCartesien-desc}

Cette classe implante les senseurs cart�siens, c'est � dire les
senseurs qui mesurent des coordonn�es cart�siennes de leur direction
cible. Ces pseudo-senseurs permettent par exemple de faire des trac�s
de passage d'astre dans des champs de vue de senseurs r�els, en
associant un triplet de tels senseur pour obtenir les trois
coordonn�es de l'astre.
\subsubsection*{interface publique}\label{sec:SenseurCartesien-int}
\begin{verbatim}
#include "marmottes/SenseurCartesien.h"
\end{verbatim}

\begin{tableFonctionsFixe}{SenseurCartesien : m�thodes publiques}
{\label{tab:SenseurCartesien-met-pub}}
{force le senseur � respecter les unit�s de mesures dans ses sorties}

\signature{\fonc{SenseurCartesien}}
          {(const string\& \argument{nom},\\
            const RotDBL\& \argument{repere},\\
            const VecDBL\& \argument{axeCalage},\\
            double \argument{precision},\\
            codeCible \argument{code},\\
            const StationCible *\argument{ptrStation},\\
            const VecDBL\& \argument{observe},\\
            Parcelle* \argument{ptrChampDeVue},\\
            Parcelle* \argument{ptrChampInhibitionSoleil},\\
            Parcelle* \argument{ptrChampInhibitionLune},\\
            Parcelle* \argument{ptrChampInhibitionCentral},\\
            double \argument{margeEclipseSoleil},\\
            double \argument{margeEclipseLune},\\
            double \argument{seuilPhaseLune},\\
            const VecDBL\& \argument{reference})
          }&

construit une instance � partir des donn�es technologiques\\

\hline

\signature{\fonc{SenseurCartesien}}
          {(const SenseurCartesien\& \argument{s})}&

constructeur par copie\\

\signature{SenseurCartesien\& \fonc{operator =}}
          { (const SenseurCartesien\& \argument{s})}&

affectation\\

\hline

\signature{\fonc{\~{}SenseurCartesien} ()}
          {}&

destructeur, ne fait rien dans cette classe\\

\hline

\signature{void \fonc{respecterMesures} ()}
          {}&

force le senseur � respecter les unit�s de mesures dans ses sorties\\

\signature{void \fonc{convertirMesures} ()}
          {}&

force le senseur � convertir les unit�s de mesures dans ses sorties\\

\hline

\signature{Senseur* \fonc{copie} () const}
          {}&

op�rateur de copie virtuel\\

\signature{void \fonc{modeliseConsigne}}
          {(const Etat\& \argument{etat},\\
            double \argument{valeur})\\
          \throw{CantorErreurs, MarmottesErreurs}}&

mod�lise la consigne \argument{valeur} dans l'\argument{etat} fourni\\

\signature{double \fonc{mesure} (const Etat\& \argument{etat})}
          {\throw{MarmottesErreurs}}&

retourne la mesure que produirait le senseur dans l'\argument{etat}
fourni\\

\end{tableFonctionsFixe}

\subsubsection*{implantation}\label{sec:SenseurCartesien-impl}
Il n'y a ni attribut priv�, ni attribut prot�g�.

Les m�thodes prot�g�es sont d�crites dans la table~\ref{tab:SenseurCartesien-met-prot}.
\begin{tableFonctionsFixe}{SenseurCartesien : m�thodes prot�g�es}
{\label{tab:SenseurCartesien-met-prot}}
{constructeur par d�faut. Il est d�fini explicitement uniquement pour }

\signature{\fonc{SenseurCartesien} ()}
          {}&

constructeur par d�faut. Il est d�fini explicitement uniquement pour
pr�venir celui cr�� automatiquement par le compilateur et ne doit pas �tre
utilis�.
\\
\end{tableFonctionsFixe}
