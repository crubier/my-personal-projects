% -*- mode: latex; tex-main-file: "club.tex" -*-
% $Id: environnement.tex,v 1.2 2005/03/11 16:27:57 chope Exp $

La biblioth�que \bibliotheque{club} a besoin, au minimum, des
biblioth�ques standards du langage \langage{c++}. Pour pouvoir acc�der
� des fichiers au format Madona ou XML, la biblioth�que
\bibliotheque{club} a besoin repectivement de la biblioth�que
\bibliotheque{madona} ou de la biblioth�que \bibliotheque{xerces}.

Le syst�me de traduction a besoin � l'ex�cution de deux variables
d'environnement, une pour sp�cifier la langue, l'autre pour sp�cifier
une liste de r�pertoires pouvant contenir les dictionnaires. Ces deux
variables sont configurables � l'installation de la biblioth�que (voir
la section~\ref{sec:installation}). Leur valeur par d�faut est
\texttt{CLUB\_LANG} pour la langue utilisateur et
\texttt{CLUB\_TRADPATH} pour la liste de r�pertoires. L'installation
\begin{changebar}
standard de la biblioth�que pour le service \textsc{sb/ms}
\end{changebar}
sp�cifie cependant des variables diff�rentes, � la fois pour des
raisons historiques et par soucis d'homog�n�it� avec l'environnement
\textsc{mercator} ; les variables sont alors \texttt{MRC\_USER\_LANG}
pour la langue utilisateur et \texttt{MRC\_USER\_TRADPATH} pour la
liste des r�pertoires.

La conception du syst�me de traduction permet � celui-ci de tourner en
l'absence de tout fichier de traduction. Dans ce cas la langue interne
du logiciel est utilis�e.

\begin{changebar}
Le support optionnel des fichiers XML a besoin � l'ex�cution d'une
variable d'environnement pour sp�cifier une liste de r�pertoires
pouvant contenir les DTD (\emph{Document Type definition}, fichiers
d�crivant les syntaxes XML adopt�es dans \bibliotheque{club}) et le
fichier des unit�s par d�faut. Cette variable est configurable �
l'installation de la biblioth�que (voir la
section~\ref{sec:installation}). Sa valeur par d�faut est
\texttt{CLUB\_XMLPATH}. L'installation standard de la biblioth�que
pour le service \textsc{sb/ms} sp�cifie cependant une variable
diff�rente, par souci d'homog�n�it� avec les deux variables
pr�c�dentes ; la variable est alors \texttt{MRC\_USER\_XMLPATH}.
\end{changebar}
