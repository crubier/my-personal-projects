% -*- mode: latex; tex-main-file: "marmottes-utilisateur.tex" -*-
% $Id: SenseurLimbe.tex,v 1.8 2004/06/21 14:43:24 marmottes Exp $
\subsection{classe SenseurLimbe}\label{sec:SenseurLimbe}

\subsubsection*{description}\label{sec:SenseurLimbe-desc}

Cette classe implante les senseurs de limbe, c'est � dire des senseurs
d'angles di�dres classiques observant obligatoirement le bord d'un
astre ayant un diam�tre apparent suffisant.
\subsubsection*{interface publique}\label{sec:SenseurLimbe-int}
\begin{verbatim}
#include "marmottes/SenseurLimbe.h"
\end{verbatim}

\begin{tableFonctionsFixe}{SenseurLimbe : m�thodes publiques}
{\label{tab:SenseurLimbe-met-pub}}
{destructeur, lib�re la m�moire allou�e aux champs d'inhibition}

\signature{\fonc{SenseurLimbe}}
          {(const string\& \argument{nom},\\
            const RotDBL\& \argument{repere},\\
            const VecDBL\& \argument{axeCalage},\\
            double \argument{precision},\\
            Parcelle *\argument{ptrChampDeVue},\\
            Parcelle *\argument{ptrChampInhibitionSoleil},\\
            Parcelle *\argument{ptrChampInhibitionLune},\\
            Parcelle *\argument{ptrChampInhibitionCentral},\\
            double \argument{margeEclipseSoleil},\\
            double \argument{margeEclipseLune},\\
             double \argument{seuilPhaseLune},\\
            const VecDBL\& \argument{referenceZero},\\
            const VecDBL\& \argument{axeSensible})
          }&

construit une instance � partir des donn�es technologiques\\

\hline

\signature{\fonc{SenseurLimbe} (const SenseurLimbe\& \argument{s})}
          {}&

constructeur par copie\\

\signature{SenseurLimbe\& \fonc{operator =} (const SenseurLimbe\& \argument{s})}
          {}&

affectation\\

\hline

\signature{\fonc{\~{}SenseurLimbe} ()}
          {}&

destructeur\\

\hline

\signature{Senseur* \fonc{copie} () const}
          {}&

op�rateur de copie virtuel\\

\hline

\signature{int \fonc{controlable} (const Etat\& \argument{etat})}
          {\throw{MarmottesErreurs}}&

indique si le senseur serait capable de contr�ler le satellite dans
l'\argument{etat} fourni\\
\end{tableFonctionsFixe}

\subsubsection*{implantation}\label{sec:SenseurLimbe-impl}
Il n'y a ni attribut prot�g� ni attribut priv�. Les m�thodes prot�g�es
sont d�crites dans la table : \ref{tab:SenseurLimbe-met-prot}.
\begin{tableFonctionsFixe}{SenseurLimbe : m�thodes prot�g�es}
{\label{tab:SenseurLimbe-met-prot}}
{pour les capteurs de limbe, on ne sait pas calculer l'�cart angulaire}

\signature{\fonc{SenseurLimbe} ()}
          {}&
constructeur par d�faut. Il est d�fini explicitement uniquement pour
pr�venir celui cr�� automatiquement par le compilateur et ne doit pas �tre
utilis�.
\\

\signature{void \fonc{ecartFrontiere}}
          {(const Etat\& \argument{etat},\\
            double *\argument{ptrEcartFrontiere},\\
            bool *\argument{ptrAmplitudeSignificative}) const\\
            \throw{CantorErreurs}}&

pour les capteurs de limbe, on ne sait pas calculer l'�cart angulaire
entre la cible et la fronti�re. Cette m�thode se contente donc de
retourner $+1$ si le limbe est visible et $-1$ s'il ne l'est pas et
d'indiquer par une valeur fausse dans la variable point�e par
\argument{ptrAmplitudeSignificative} que la valeur num�rique est non
significative.\\
\hline
\end{tableFonctionsFixe}
