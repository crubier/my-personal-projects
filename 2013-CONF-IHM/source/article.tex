%
% Ce document est écrit au format UTF-8.
% Il est adapté depuis http://www.sigchi.org/publications/chipubform.
%
\documentclass{ihm}

% Décommentez cette ligne pour remettre les numéros de page.
% \pagenumbering{arabic}


% Quelques packages indispensables.
\usepackage{balance}
\usepackage{graphics}
\usepackage{times}
\usepackage{url}

% Style pour les URLs
\makeatletter
\def\url@leostyle{%
  \@ifundefined{selectfont}{\def\UrlFont{\sf}}{\def\UrlFont{\small\bf\ttfamily}}}
\makeatother
\urlstyle{leo}


% Ne pas modifier
\def\pprw{21cm}
\def\pprh{29.7cm}
\special{papersize=\pprw,\pprh}
\setlength{\paperwidth}{\pprw}
\setlength{\paperheight}{\pprh}
\setlength{\pdfpagewidth}{\pprw}
\setlength{\pdfpageheight}{\pprh}

% Hyperref doit être le dernier package à inclure.
% N'oubliez pas de renseigner les informations qui se trouvent ici.
\usepackage[pdftex]{hyperref}
\hypersetup{
pdftitle={Format des publications finales pour IHM'13},
pdfauthor={LaTeX},
pdfkeywords={format, instructions, qualité, actes de conférence},
bookmarksnumbered,
pdfstartview={FitH},
colorlinks,
citecolor=black,
filecolor=black,
linkcolor=black,
urlcolor=black,
breaklinks=true,
}

% Utilisez \tabhead{} pour les en-têtes de tableaux.
\newcommand\tabhead[1]{\small\textbf{#1}}


% Début du document
\begin{document}

\title{Un langage formalisable pour la définition d'applications interactives
  embarquées et critiques}

\numberofauthors{3}
\author{
  \alignauthor Vincent Lecrubier\\
    \affaddr{ONERA}\\
    \affaddr{2 Avenue Edouard Belin}\\
    \affaddr{31400, Toulouse, France}\\
    \email{Vincent.Lecrubier@onera.fr}
  \alignauthor Bruno d'Ausbourg\\
    \affaddr{ONERA}\\
    \affaddr{2 Avenue Edouard Belin}\\
    \affaddr{31400, Toulouse, France}\\
    \email{Bruno.d.Ausbourg@onera.fr}
  \alignauthor Yamine Aït-Ameur\\
    \affaddr{INPT/ENSEEIHT}\\
    \affaddr{2 rue Charles Camichel}\\
    \affaddr{31000, Toulouse, France}\\
    \email{yamine@enseeiht.fr}
}

\maketitle

\begin{abstract}
Cet article décrit et utilise le format de mise en page à respecter pour les
soumissions à IHM'13. Après acceptation, ces soumissions seront publiées dans
les actes d'IHM'13, ainsi que sur l'ACM Digital Library, sauf les communications
informelles qui seront publiées dans un volume annexe séparé. Comme certains
détails ont changé par rapport aux années précédentes, nous vous prions de bien
lire ce document même si vous avez déjà soumis à IHM auparavant.
\end{abstract}

\keywords{
	Format; instructions; qualité; actes de conférence; les mots-clés doivent
	être séparés par des points-virgules.
}

\category{H.5.m.}{Information Interfaces and Presentation (e.g. HCI)}{Miscellaneous}.
Voir \url{http://www.acm.org/about/class/1998/} pour la liste complète des catégories ACM.

\section{Introduction}

Méthodes  et  techniques de  descriptions  formelles  ont été,  depuis
longtemps, appliquées au  domaine des des systèmes  critiques pour des
besoins  de  sûreté.  C'est   particulièrement  le  cas  des  systèmes
aéronautiques  dont le  fonctionnement peut  mettre en  jeu la  vie de
passagers.   De  nombreux travaux  de  recherche  se sont  attachés  à
élaborer et proposer des moyens pour concevoir, programmer et vérifier
les   fonctionnalités    de   systèmes   critiques    en   particulier
aéronautiques.   Ces  systèmes et  les  processus  qui encadrent  leur
développement sont par ailleurs  l'objet d'étapes de certification qui
établissent que  le produit  et le  développement qui  a conduit  à le
construire  sont  en conformité  avec  des  standards établissant  des
règles précises les définissant, tels  la DO178C \cite{DO178C} pour le
logiciel aéronautique embarqué.

Les systèmes  d'interaction homme-machine (IHM) n'ont  bénéficié ni de
la  même attention  ni  du  même effort  de  recherche.  Pourtant  des
systèmes d'interaction  mettant en oeuvre des  capacités électroniques
et numériques hautement sophistiquées ont fait leur apparition au sein
de cockpits d'avions de nouvelle  génération.  Ces systèmes se fondent
sur  des  logiciels à  la  fois  volumineux  et complexes  et  doivent
répondre  à des  contraintes  sévères en  matière d'utilisabilité,  de
sécurité  ou  de   sûreté.   Ces  logiciels  mettent   en  oeuvre  des
applications dont le comportement, compte tenu de leur criticité, doit
se  conformer avec  un  niveau très  élevé d'assurance  à  ce qui  est
attendu  du système.  En effet  une  erreur dans  l'un des  composants
d'interaction logiciels peut conduire à  une erreur humaine ou système
dont les effets peuvent se révéler catastrophiques.

Le récent rapport du Bureau  d'Enquêtes et d'Analyses \cite{BEA12} sur
le crash  de l'Airbus A330  d'Air France  lors du vol  AF447 Rio-Paris
pointe  un réel  dysfonctionnement du  système d'interaction,  et plus
particulièrement du  directeur de  vol, ayant affiché  des indications
qui n'auraient pas dû l'être et ayant ainsi conduit le pilote à cabrer
son appareil alors que ce dernier était déjà en train de décrocher.

Le  coeur du  problème réside  essentiellement  dans le  fait que  les
processus  standards  de   développement  des  systèmes  aéronautiques
critiques se fondent la plupart du  temps sur des cycles linéaires (du
type des  cycles en V  ou de ses variantes)  qui s'avèrent peu  ou mal
adaptés   à  la   conception   et  au   développement  des   logiciels
d'interaction qui  requièrent des  cycles d'itérations  impliquant les
usagers et mêlant opérations de test  et phases de conception dans les
processus mis en oeuvre. Une autre difficulté, propre au développement
des  systèmes  d'interaction,  est  imputable à  la  multiplicité  des
communautés techniques  et scientifiques intervenant dans  le champ de
la définition des systèmes d'IHM.  Manquent à ces communautés diverses
un  langage et  des notations  qui  soient, sinon  communes, du  moins
intelligibles par toutes. Dès lors le traitement de la sûreté et de la
sécurité  de ces  systèmes  d'interaction critiques  ne  fait appel  à
aucune  méthode ni  aucun outil  spécifiques.  Dans  ce contexte,  les
processus de validation  de ces systèmes demeurent pauvres  car ils se
limitent pratiquement à des phases  intensives de test et d'évaluation
en fin  de processus de développement,  qui plus est sans  disposer de
réelle  référence, en  particulier  formelle, vis  à  vis de  laquelle
confronter le système développé.

Ce  papier  suggère d'introduire  cette  référence  formelle lors  des
étapes  amont  des  processus  de  développement  des  systèmes  d'IHM
critiques  en  proposant un  langage,  formalisable,  qui permette  la
description des  applications interactives,  tant du  point de  vue de
leur présentation que de leur  comportement, ainsi que l'expression d'
exigences des concepteurs concernant  l'interaction encodée au sein de
ces  applications.   Ce langage  veut  être  suffisamment abstrait  et
convivial pour être compris et utilisables par tous les participants à
la conception  et au développement  de l'IHM. Il veut  également être,
sinon formel,  du moins  formalisable de telle  sorte que  qu'un texte
rédigé  dans   ce  langage  puisse   être  soumis  à   des  opérations
automatiques  ou  semi-automatiques  de vérifications  et  de  preuves
formelles, de génération de tests  pertinents du système à développer,
et de génération de code plus  sûr. De manière à accroître l'assurance
que le logiciel développé se comporte bien comme prévu.

\section{Travaux connexes}
\label{sec:travaux}

La formalisation des systèmes d'interaction a pourtant fait l'objet de
travaux  importants  lors  de  ces dernières  années:  bon  nombre  de
modèles, de langages ou de méthodes  formels ont vu le jour pour aider
à spécifier, concevoir, vérifier et implanter ces systèmes.

Une première classe  de modèles basés sur les  systèmes de transitions
et plus particulièrement sur  les extensions d'automates d'états finis
et de  réseaux de Petri ont  utilisé les automates d'états  finis pour
décrire  des   comportements  interactifs  de  base.    De  manière  à
circonvenir  les  problèmes liés  à  l'explosion  combinatoire de  ces
modèles  de  nombreuses extensions  ont  été  proposées. Les  machines
d'états  hiérarchiques  ou  les statecharts  décrivent  des  automates
composables  facilitant  la  spécification de  systèmes  complexes  et
permettant la réutilisation  des modèles. Les automates  à états finis
peuvent  être  utilisés  dans   le  cadre  d'outils  de  programmation
\cite{Blanch02,Blanch06}   ou  peuvent   être   associés  à   d'autres
formalismes      pour      décrire       d'autres      aspects      de
l'interaction. En \cite{Jacob99} les  automates modélisent les apsects
de l'interaction liés à des entrées  discrètes tandis que les flots de
données représentent les entrées continues. 

Le modèle  de capteurs formels  \cite{Accot97} utilise les  réseaux de
Petri de haut niveau  \cite{Jensen95} pour décrire les transdormations
successives  d''événements  d'entrée. Les  réseaux  de  Petri ont  été
utilisés  par  plusieurs  méthodes  et  techniques  pour  décrire  les
systèmes  d'IHM.  Le  formalisme  ICO \cite{palanque93,palanque97}  se
fonde sur  une approche orientée  objet pour modéliser  les composants
statiques d'un système d'interaction et  sur les réseaux de Petri pour
en  décrire  la  dynamicité  :  PetShop  \cite{Bastide02,Schyn03}  est
l'outil basé sur ICO.

D'autres descriptions  formelles des  systèmes interactifs  se fondent
sur la  notion d'interacteurs.  Les interacteurs  peuvent se concevoir
comme  des  entités  autonomes et  communicantes  interagissant  entre
elles, de  manière interne au système  d'interaction, ou interagissant
de façon  externe avec le système  interfacé ou l'usager par  le biais
d'événements (ou de stimuli). Les deux premiers modèles d'interacteurs
développés le  furent dans le  cadre du  projet AMODEUS. Le  modèle de
York \cite{york93}  fournit un cadre  pour la description  de systèmes
d'interaction à l'aide de notations  orientées modèles telles que Z ou
VDM.   Le  modèle  du  CNUCE  \cite{Paterno92}  utilise  quant  à  lui
LOTOS. Des interacteurs formels ont également été décrits en utilisant
le  langage synchrone  Lustre :  \cite{nousdsvis96} montre  comment de
tels interacteurs peuvent être dérivés d'une description informelle du
système d'interaction. Ces interacteurs  formels peuvent être composés
pour constituer  un modèle  formel de l'IHM  et peuvent  être utilisés
pour vérifier le système conçu  ou générer des tests \cite{dsvis98} et
pour, plus  généralement, valider  le système  d'interaction développé
\cite{ausbourg98}. 

\cite{yamine98a,yamine98b} introduisent  une approche  modulaire basée
sur  l'usage  de  la  méthode  B.   Les  spécifications  d'un  système
interactif opérationnel  y sont formalisées en  assurant qu'un certain
nombre   de   propriétés   ergonomiques   sont   préservées   par   le
système. Diverses formes d'interaction  ont ainsi été formalisées dans
le cadre  de ces travaux  en permettant la vérification  de nombreuses
propriétés de  ces systèmes telles leur  observabilité, leur honnêteté
ou leur robustesse.  Le travail  présenté en \cite{yamine00} montre la
description  et la  preuve, à  l'aide de  B, de  toutes les  étapes de
raffinement  d'un  modèle  abstrait  de  système  jusqu'à  son  codage
concret.

Le  grand  mérite  de  ces  travaux est  ainsi  d'avoir  démontré  que
l'interaction peut être formalisée et est donc formalisable. Toutefois
ces  approches   construisent  leurs  modèles  directement   dans  les
formalismes cibles.  Cela  suppose la maîtrise de  leurs notations. Or
les  différents intervenants  dans le  cadre de  la définition  de ces
systèmes ne  sont jamais  des praticiens de  ces formalismes  qui, dès
lors, demeurent inutiles et sans grand intérêt pour eux. Il était donc
nécessaire  de construire  un  langage utilisable  ayant toutefois  la
capacité d'offrir les mêmes services.

\section{Le besoin auquel nous voulons répondre} 

Le langage proposé vise donc à répondre à différents besoins
exprimés par beaucoup d'acteurs dans le domaine de IHM
embarquées critiques.

Le premier de ces besoins est de pouvoir vérifier certaines
proprietés de sûreté, le plus tôt possible dans le cycle de
développement. A l'heure actuelle, les proprietés de sûreté ne
sont vérifiées dans le meilleur des cas qu'après une certaine
concrétisation du concept d'IHM, et dans le pire des cas
qu'après l'implémentation complète de l'IHM, lors d'une campagne
de tests. Certaines erreurs n'apparaissent que lors de la
validation finale, faisant perdre beaucoup de temps et d'argent,
alors qu'elles auraient pu être corrigées dès les premières
étapes de la spécification. Il manque donc un moyen de vérifier
formellement des proprietés de sûreté dès les premières phases
d'un projet, quand celui ci est encore abstrait.

Le second besoin provient de l'enrichissement actuel du monde de
l'IHM. Ce monde vit actuellement une phase de grande
diversification, et de nombreux horizons s'offrent aux
concepteurs en termes d'interaction. Cependant, en l'absence
d'outils permettant de travailler sur les aspects conceptuels
des IHM, les concepteurs se mettent rapidement à raisonner en
termes d'implémentation concrète, en se basant sur des notions
qu'ils connaissent déjà. Ainsi, ils passent trop rapidement du
besoin à la solution, négligeant d'envisager les solutions les
plus récentes et plus adaptées à leur besoin. Il manque donc un
moyen de travailler efficacement sur les aspects conceptuels des
IHMs, permettant de se poser les bonnes question en termes
d'interaction.

Le troisième besoin est d'ordre humain. Les différents acteurs
travaillant à la conception d'une IHM ont tous des métiers
différents. On trouve pêle mêle, des designers, des
programmeurs, des ergonomes, des ingénieurs systèmes, ainsi que
des spécialistes en facteurs humains ou en architecture
logicielle. Ces différents métiers ont des cultures variées, et
donc assez peu de points communs. Ces barrières rendent parfois
la communication peu efficace, compliquant le processus de
conception des IHM. Il est donc important d'offrir un langage
commun à ces différents acteurs afin de fluidifier leur
collaboration.

Enfin, le dernier besoin, lié au précédent, est dû à l'absence
de lien formel entre les différents artefacts générés par les
différents acteurs. La conception d'une IHM critique passe
généralement par la réalisation de produits intermédiaires :
documents de spécification, documentation de cas de tests,
différents prototypes, code d'implémentation... Ces différents
documents ont un lien assez laxe et informel entre eux, ce qui
pose des problèmes de traçabilité et de synchronisation entre
les différentes étapes du processus. Il y a donc clairement
besoin d'un document pivot entre ces différents domaines.

Le langage proposé est développé dans le but de répondre au
mieux à ces différents besoins. Ces différents besoin sont
parfois contradictoires. Par exemple, pour pouvoir être compris
par tous les acteurs, le langage doit être assez simple, ce qui
s'oppose au besoin d'expressivité, nécéssaire pour pouvoir
vérifier des proprietés formelles intéressantes. Notre travail
consiste donc à trouver la solution présentant le meilleur
compromis face à tous ces besoins.

\section{Le langage} 

Le langage proposé s'articule autour de la notion d'Interacteur.
Les interacteurs sont composés : un interacteur peut être
composé de sous-interacteurs. Ainsi, une IHM complète sera
représentée sous la forme d'un interacteur, composé de ses sous
éléments : fenêtres, contrôles, sons... qui sont eux même des
interacteurs.

Le langage vise à modéliser une interface dite abstraite. Cela
signifie que l'on ne modelisera, au moins dans un premier temps,
que les aspects conceptuels de l'IHM, sans s'occuper de leur
implémentation réelle sur le produit fini. Ainsi, l'on pourra
abstraire différents types d'entités sous un même concept, ce
qui simplifie le langage et son utilisation. Par exemple, les
bien connus boutons à deux états (ToggleButton) et case à cocher
(CheckBox) ont une même fonction : permettre à l'utilisateur de
communiquer une valeur booléenne à la machine; ils sont donc
regroupés sous le concept d'entrée booléenne (BooleanInput).
Tous les éléments d'IHM sont ainsi regroupés dans différentes
catégories représentant les briques de bases de l'interaction
homme-machine.

Les types d'interacteurs de base du langage sont ainsi en nombre
relativement réduit, tout en permettant de prendre en compte
l'intégralité des interactions homme-machine. On trouvera des
entrées et sorties de textes, réels, entiers, booléens, choix de
sous ensembles parmi un ensemble, ainsi que de déclenchement
d'évenements. La nature du langage permet de combiner ces
briques de base afin de définir des interactions plus complexes.

On l'a vu, un interacteur se compose de sous interacteurs. Ceci
permet de représenter la structure statique des IHM. Afin
d'exprimer le dynamisme des IHM, un interacteur sera aussi
composé d'une partie exprimant sont comportement. Le
comportement des interacteurs sera exprimé à l'aide d'une
machine à états. L'interacteur comprendra ainsi un ensemble de
variables d'état et un ensemble de règles définissant
l'évolution des ces variables d'état, en fonction des évenements
reçus par l'interacteur par le biais de ses sous-interacteurs.

Ces règles d'évolution de l'état de l'interacteur permettent
d'exprimer plusieurs types de comportements.

Un premier type de comportement représente l'état initial de
l'interacteur et des sous interacteurs. Il permet d'exprimer des
proprietés qui seront vraies à l'initialisation de
l'interacteur, sans forcément le rester par la suite. Par
exemple, à l'ouverture d'une page web, la vue est placée tout en
haut de la page.

Un autre type de comportement est défini comme un invariant. Il
représente des proprietés qui resteront vraies à tout instant
lors de l'exécution. Ainsi, l'heure affichée dans la barre
d'état d'un smart phone, est à tout moment égale à l'heure telle
que se le représente le système.

Un dernier type de comportement est défini comme la réponse à un
évenement reçu. Il spécifie que si l'interacteur reçoit un
certain évenement, et que certaines conditions (garde de la
transition) sont réunies, alors un changement d'état aura lieu.
Par exemple, si l'on clique sur le bouton "quitter", la fenêtre
se fermera.





\small
\bibliographystyle{acm-sigchi}
\bibliography{bruno}


\end{document}
