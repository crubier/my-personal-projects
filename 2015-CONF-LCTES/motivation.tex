%-----------------------------Description du contenu--------------------------- 

%Crash  analyses  reveal  1)  a  lack of  security  in  human  machine
%interaction (e.g.  AF  447) 2) the interactive systems  do not behave
%as intended  %\textcolor{blue}{ Nowadays, compared to  other critical
%domains   (command/control   in  aeronautics,   nuclear,   automotive
%industries ...  )   there is no tool, technique,  method nor language
%to describe properly what is intended with respect to interaction.  {\bf A DEVELOPPER } 

%Dire que l'on ne  peut pas faire d'analyse sur les  800 000 lignes de
%codes d'ihm  car les techniques  développées sur les  autres systèmes
%are not relevant for interactive systems.  

%1- Une  description abstraite du comportement  en termes d'interation
%(entrée  usager,  état  et  reponse  système) de  ce  qui  doit  être
%embarquer dans ces 800 000  lignes est nécessaire.  Cette description
%peut me servir à vérifier que le code implante bien les comportements
%abstraits attendus et  mieux si je peux faire  cette vérification, on
%peut produire un code sûr qui comportera ces comportements.  

%There is a need of a language which 1) abstract the main interactive concepts and } 

%So, describing intended behaviors is central for ensuring security of interactive systems.  

%Need of resources for that on interactive systems.  .
%-----------------------------------------------------------------------------
A lot  of research  work have  focused on how  to design,  program and
verify functional concerns for  critical systems and more particularly
aeronautical  systems.  HMI   systems  did   not   benefit  from   the
same  attention   and efforts. 

\subsection{Context}
%------------------------------------------------
%\begin{itemize}
%\item Critical embedded systems 
%\item ARINC 661 and development practices A380 cockpits. Client server with its design constraints.
%\item static description of screen displays, of interactive devices, layout, graphical representations etc. 
%\item Traditional approaches in industry are based on a posteriori  blind testing. examples are A380
%\item Diversity of stakeholders and of involved disciplines and communities (human factors, system designers, ergonomists, software developers, graphical designers) 
%\end{itemize}

%In fact, the  safety concern, while being acknowledged  as specific in
%the design  stages of  the development  processes of  aeronautical HMI
%systems, is not really dealt with  specific methods and tools. In this
%context, the  validation process of these  interactive applications is
%very restricted and  poor because it resides practically  only in test
%and evaluation phases at the end of the development process.  Moreover
%there is no actual formal reference  to check the implementation is in
%conformance with.
%------------------------------------------------

A significant  amount of work has  focused on devising models  for the
development  process of  software  systems in  the  field of  software
engineering.

The system development  process in critical domains  as, for instance,
in aeronautics  inherited these  models.  This  process is  now widely
based on  the use of standards  that take into account  the safety and
security  requirements   of  the   systems  under   construction.   In
particular the DO178C  standard~\cite{DO178C}, in aeronautics, defines
very strict  rules and instructions  that must be followed  to produce
software  products,  embedded  systems   and  their  equipments.   The
objective is to ensure that the  software performs its function with a
safety level in accordance with the safety requirements.

The HMI development does not  follow the same processes. Nevertheless,
in aeronautics, HMI  systems are now made up by  multiple hardware and
software components embedded in  aircraft cockpits.  These systems are
large and complex artifacts that  also face tough constraints in terms
of  usability,   security  and   safety.   They   support  interactive
applications  that must  behave  as  intended with  a  high degree  of
assurance  because  of their  criticity.   An  error in  the  software
components that implement interactions  in these applications may lead
to a human or system fault that may have catastrophic effects.  

For example, the BEA report \cite{BEA12} about the crash of Rio-Paris AF
447 A330 Airbus establishes that,  during the flight, interface system
displayed some actions to be performed by the pilot in order to change
the pitch  of the aircraft  and to nose it  up while it  was stalling.
These indications should clearly not  have been displayed.  Indeed, by
following those  erroneous displayed instructions the  pilot increased
the stalling of the aircraft.

In  fact,  in  the  industrial context,  the  development  process  of
critical interactive  embedded applications stays very  primitive. The
usual  notations  are  essentially  textual and  coding  is  generally
performed  from  scratch or  by  reusing  previous developments  based
themselves  on textual  specifications. In  aeronautics, the  produced
code must be in  conformance with the ARINC~661 standard~\cite{ARINC}.
It may  be noticed that  some tools  recently appeared to  enhance the
design and  coding stages of these  systems.  But these tools,  as for
instance   Scade   Display~\cite{scade-display},  deal   mainly   with
presentation layers of the systems and  do not deal with their complex
functional behaviour. In  this context, the validation  process of the
interactive  applications  is  very  restricted and  poor  because  it
resides practically  only in  a massive test  effort and  in expensive
evaluation phases  at the  end of  the development  process.  Moreover
there is no actual formal reference  to check the implementation is in
conformance with. So new approaches and new paradigms are today needed
to help  in the development  process of critical  interactive embedded
applications.

\subsection{Requirements}

We  believe that  part  of these  issues  are  due to  the  lack of  a
well-defined language for representing  interactive software design in
a way  that allows, on  the one hand,  system designers to  iterate on
their designs before  injecting them in a development  process and, on
the other hand, system developers  to check their software against the
chosen  design. Such  a hub  language  (similar to  VHDL for  hardware
description and Scade for safety-critical control and command software
development)  would bring  increased  flexibility  in the  development
process leading not  only to easier iterations within  and between its
different phases but  also to the automation of parts  of the process.
Some requirements can be expressed concerning such a language.

\begin{itemize}
\item \textbf{REQ1.}  The language is  a domain specific  language for
HMI systems of embedded systems. It  permits to use and manipulate the
HMI  design  concepts and  so  to  describe properly  the  interactive
behaviors the software will implement. In particular the language must
permit  to  describe  and  to  handle  both  continuous  and  discrete
interactions.
\item  \textbf{REQ2.}The language  is a  pivot language  that must  be
read,  understood and  written by  different stakeholders  coming from
different scientific disciplines
\item \textbf{REQ3.}  The language is  formal. Its semantics  is clear
and unambiguous.
\item  \textbf{REQ4.} Descriptions  in this  language may  be used  to
generate a safe code for an  interactive application and this code may
be compatible with the ARINC 661 Standard \cite{ARINC}.
\item \textbf{REQ5.}  The language permits  to design in the  same way
any  interactive  component  of  an  interactive  system  through  the
desciption of its input, output, and internal states.
\item \textbf{REQ6.}  The language permits  to devise and  to describe
complex interactive  systems by giving some  syntactical constructions
to assemble and compose interactive components.
\end{itemize}

We devised such a language  (the LIDL Interaction Definition Language)
to  help  the definition  of  critical  interactive applications.  The
following section presents the LIDL language.